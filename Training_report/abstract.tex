\newpage
\thispagestyle{empty}

\begin{center}
\hue{\LARGE Abstract}\\[1.5cm]
\end{center}
\begin{flushleft}
I was selected as a student developer \cite{b1} for The Honeynet Project Organization under Goolge Summer of Code 2020. In this project, my main focus was on improving the working functionality of one of the Web based High interaction Honeypot.

\setlength{\parindent}{3ex}
The Honeypot is divided into two part, one is called SNARE \cite{b2}, this is considered as the frontend of the Honeypot. SNARE is responsible for cloning the given website and then uses those cloned webpages to mimick the actual working website. This can gives attacker a feeling of read website. The another part is called TANNER \cite{b3}, its the brain of the honeypot. TANNER captures all the data that is being generated by the malicious actor. It then performs analysis on that data to make give idea about who the attacker was and what kind of vulnerability were they trying to exploit.\\

\setlength{\parindent}{3ex} 
Specifically, I worked on improving the SNARE's ability to clone and serve the pages. Also, I added support for serving pages with TLS. On TANNER's side, I added PostgreSQL support to help tanner maintain sessions for the long term. Along with this I created small script to help users migrate from the old DB setup, which was using Redis as the main DB, to a new database setup which used PostgreSQL along with Redis.\\

\setlength{\parindent}{3ex}
Other than that I made small changes to fix existing bug in the code base. Also, I added support for different kind of Injection templates which improve the TANNER's capabilities to mimick the attack resulting in better information capture.
\end{flushleft}
\vfill