\newpage
\thispagestyle{empty}
\chapter{SOFTWARE TRAINING WORK UNDERTAKEN} \label{swtraining}
The SNARE & TANNER was developed by Lukas Rist back in 2016. Since then this project is going under continuos development. The development is mostly done by various students selected under Google Summer of code. For GSoC 2016 Ms Evengeniia Tokarchuk was selected as the student developer and she worked under the guidance of Lukas Rist to add the initial features of the Honeypot.

In 2016, it was decided that using a memory based database would be a good choice. This was considered because at the very beginning of the project high speed performance was required. Also since the Honeypot was still in early phases of development not a lot of data was being generated by it. So storing less amount of data on RAM(memory) wasn't an issue. But as the project grew over the years new modules were added to it along with new modules came large amount of data which was being logged by the honeypot.

Now the issue with the large amount of data was that for storage in-memory based database, Redis, was being used. Redis provide speedy access to the data by keeping all the data on RAM, hence \textit{in-memory database}. So if the honeypot is being setup on a low spec server/system then that system would crash because of running out of memory(RAM) after certain point. And once the server crashes it has to be restarted manually, which kind of defeats the point of Honeypot.

That is why this year I worked on improving the storage functionality. Since in this project I used both PostgreSQL and REDIS in a combination to provide a better working and storage model to the Honeypot. I learned about the issues REDIS DB can raise and also learned the benenfit of using ORM with some major database like PostgreSQL or MySQL.
\vfill